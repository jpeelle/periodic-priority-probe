
%!TEX TS-program = xelatex

\documentclass[letterpaper,oneside,11pt,article, portrait]{memoir}
\usepackage[pdftitle={Scientific Paper Checklist}, pdfauthor={Jonathan Peelle}, colorlinks=true, urlcolor=blue]{hyperref}
\usepackage{colortbl}
\usepackage{color}
\usepackage{multirow}

\setlrmarginsandblock{.8in}{.8in}{*}
\setulmarginsandblock{.8in}{.8in}{*}

\usepackage{array,ragged2e}
\usepackage{fontspec,xunicode}
\defaultfontfeatures{Mapping=tex-text}


\definecolor{gray}{rgb}{0.6,0.6,0.6}
\newcommand{\doi}[1]{doi:\href{http://dx.doi.org/#1}{#1}}

\setsecnumdepth{section}

% For tabularx columns (memman.pdf p. 233-234
%\renewcommand{\tabularxcolumn}[1]{\centering{p{#1}}}

\setromanfont{Helvetica}
\checkandfixthelayout	% for memoir class

\begin{document}
\pagestyle{empty}


\begin{centering}

{\Large Periodic Priority Probe}

\small{\url{http://github.com/jpeelle/periodic-priority-probe}}

\end{centering}

\vspace{.2in}

\noindent \begin{tabularx}{\textwidth}{ l X l X}
Name:& &Date:& \\
\hline
\end{tabularx}

\vspace{.2in}

\noindent {\itshape Think through all of your activities and sort them into List 1 and List 2. Use List 3 to think about what's missing, and List 4 to come up with concrete steps to change your life. Repeat at least every 6 months.}

\subsection{List 1: What are the things I am doing that I would like to keep doing?}

\begin{tabularx}{\textwidth}{>{\centering\small}X | >{\centering\small}X}
Things I'd like to keep the same& Things I'd like to do more of\\
\end{tabularx}


\vspace{1.3in}

\subsection{List 2: What are the things I am doing that I want to {\itshape stop}?}

\begin{tabularx}{\textwidth}{>{\centering\small}X | >{\centering\small}X}
Things I'd like to do less of& Things I'd like to stop completely\\
\end{tabularx}

\vspace{1.3in}

\subsection{List 3: What are the things I am not doing but would like to {\itshape start}?}
\vspace{1in} % <-- smaller on purpose --- don't add too much! You don't have time.

\subsection{List 4: Over the next 6 months, what concrete steps will I take to lengthen List 1, shorten List 2, and/or move items from List 3 to List 1?}
\small{\itshape{(Even one concrete step that actually works can make a huge difference!)}}
%\vspace{1.5in}


\end{document}